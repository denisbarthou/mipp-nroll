%===========================================================
%                              Choix de thématique
%===========================================================
% Une des quatre options 'parallelisme', 'architecture', 'systeme' 
% 'tempsreel' doit être utilisée avec le style compas2021
\documentclass{article}

\usepackage{listings}
\usepackage{algorithmicx,fullpage}
\usepackage[noend]{algpseudocode}
\usepackage{algorithm,hhline}
\usepackage{graphicx}
\usepackage{wrapfig}
\usepackage{subcaption}
\usepackage{color}

\definecolor{light-gray}{gray}{0.80}
%===========================================================
%                               Title
%===========================================================
\definecolor{dkgreen}{rgb}{0,0.6,0}
\definecolor{gray}{rgb}{0.5,0.5,0.5}
\definecolor{mauve}{rgb}{0.58,0,0.82}

\lstset{frame=tb,
  language=C,
  aboveskip=3mm,
  belowskip=3mm,
  showstringspaces=false,
  columns=flexible,
  basicstyle={\small\ttfamily},
  numbers=none,
  numberstyle=\tiny\color{gray},
  keywordstyle=\color{blue},
  commentstyle=\color{dkgreen},
  stringstyle=\color{mauve},
  breaklines=true,
  breakatwhitespace=true,
  tabsize=3
}
\lstset{emph={%  
    in%
    },emphstyle={\color{red}\bfseries}%
}%
\begin{document}
\bibliographystyle{plain}

\lstset{
  language=C,
  basicstyle=\tiny\ttfamily,
%  numbers=left,
%  xleftmargin=2em,
  frame=single,
%  framexleftmargin=2em,
  escapeinside={(*}{*)}
}

\title{Exploring Vectorization and unrolling with SMT}

%\affiliation{\institution{Bordeaux Institute of Technology} \city{Bordeaux} \country{France}}
%\affiliation{\institution{Bordeaux Institute of Technology} \city{Bordeaux} \country{France}}
\author{Denis Barthou \and Edgar Baucher \and Ahmed-Manaf Dhamani \and William Jalby}
%\affiliation{\institution{University of Versailles Saint Quentin} \city{Versailles} \country{France}}


%\date{\today}

\maketitle



\section{Introduction}
High performance kernels require multiple optimizations, that can be performed by the compiler for many of them, and by the expert user for some of them. We do not
focus particularly on codes from libraries, but rather on kernels from
applications. One of the objectives is to study how to generate
efficient source code, while still keeping the code portable,
understandable by users and allowing composition. As a counterpart, we
may not wish to reach top performance but this would be suficient to
stay close to it (let us say 80\% of the top performance) if this
allows portability, maintenance and composability.

Many optimizations at the loop level and targeting cache/memory
hierarchy are well known and can be handled by dedicated compiler
optimizations. Let us focus here on code optimizations for codes that
run in a tile fitting in cache (or that does not have
reuse) and we assumed the code has been SIMDized by the user, resorting to a library such as the C++ MIPP
wrapper \cite{mipp} in order to ensure portability accross multiple architectures. We therefore focus on a very narrow optimization
space at compiler level, namely instruction
scheduling, loop transformations (within the tile) and
register allocation, all of them having been the subject of many related works in the past. For matrix product for instance, ATLAS \cite{atlas}, published 21 years ago, proposed automatic tuning of matrix product codes, including register blocking and loop transformations, but some initial work \cite{JalbyM86} tackles the matrix multiplication hand-tuning even 15 years before ATLAS. Since then, hardware mechanisms such as  out of order execution, register renaming have been expected to mitigate the effect of  instruction scheduling and register blocking  optimizations. This study has therefore the following objectives:
\begin{itemize}
\item Determine if there is any impact still of instruction scheduling at the source level on performance 
\item Determine if there is a simple criterion deciding the unrolling factor of a code, and if it is portable accross architectures, compilers
\end{itemize}


The
particularities of the source code kernels we consider are listed here:
\begin{itemize}
  \item Vector instructions will be parametrized by their length VL (expressed in elements) so the code should be generic and cover all possible Vector Lengths
\item All instructions will be not differentiated with respect to DP versus SP. 
\item All arithmetic instructions will operate between registers (à la ARM): no operand coming from memory can be used in an arithmetic instruction. The compiler still has the possibility to rewrite a load and a computation into a computation using an address. 
\item We will use a Vector Reduction (denoted VRED) instruction capable of summing all of the components of a vector register. Such an instruction is available in ARM ISA but currently it requires a chain of between 4 and 6 instructions on X86
\item The code to be duplicated (through unrolling) will be enclosed between curly brackets. 
\item The instruction selection for vectorized code is assumed to be done by the user through MIPP. Indeed, while efficient instruction selection methods exist in current compilers, this is still limited to linear algebra codes. MIPP (or existing alternatives) removes this issue by proposing a wider access to intrinsics SIMD instructions.
\end{itemize}

\section{Matrix product, stored row-wise}
The matrix code is a kernel for libraries, but still there may be occurences of special matrix sizes that are not perfectly addressed by libraries. We will start with small rectangular bloc sizes first, multiplying a matrix of size $BLOCKI\times BLOCKK$ by a matrix of size $BLOCKK \times BLOCKJ$. We provide to the compiler a very naive code, where the loop are ordered i,j,k, the vectorization is explicit through MIPP, and unrolling is achieved in the source code annotations. The objective is to compare the behavior of the different unrolling factors, and observe how it affects register blocking In the code presented in Fig.\ref{fig:gemm}

The unrolling on k
produces a sequence of FMAs. This does not improve ILP for FMA but
this enhances register reuse and can help to hide other instruction
latencies (loads).  There is no tail code on purpose, the sizes BI,
BJ, BK are adjusted to the unrolling factors, while still keeping the
global size of matrices A,B and C within 5\% of the chosen global
size, for all unrolling factors.  The unrolling factors from 1 to 15
have been explored to evaluate the capacity of the compilers to
allocate registers and reschedule. We have limited the exploration of
the register block size $UI*UJ+UI*UK+UJ*UK$ so that it does not exceed
twice the amount of SIMD registers (so 64 for AVX512, 32 for AVX), and
$UJ*VL$ should not exceed the size of the dimension BJ. Note that usually the number of physical registers exceed by more than a factor of two the number of virtual registers. The type of the elements are defined by \texttt{TYPE} . The above code
is written using a template mechanism that is instantiated
into C++ code for all unrolling factors.  The first machine of the
evaluation is an Intel Cascade Lake 6230R with AVX-512. The compilers
evaluated are, with flag -O3: icc version 19.1.3.304,  Clang 15 with 3 register allocation methods, greedy
(base), basic and PBQP, and GCC version 11.9. icc version 2021
(dpc++ OneAPI) is based on Clang 15 and does not exhibit any differences with Clang 15 greedy. 


\begin{figure}
  \begin{lstlisting}
for (int i=0;i<BLOCKI;i+=ui) {                
  for (int j=0;j<BLOCKJ;j+= uj*nv) {           
    { ii in [0,ui],jj in [0,uj]:
        mipp::Reg<TYPE> c_ii,jj;
        c_{ii,jj}.load(&C[(i + ii )*BLOCKJ + j + jj *nv]);
    }                
    for (int k=0;k<BLOCKK;k+= uk ) {            
      { ii in [0,ui],kk in [0,uk]:
          mipp::Reg<TYPE> a{{ ii }}{{ kk }};
          a_{ii,kk} = mipp::set1<TYPE>(A[(i + ii )*BLOCKK+k+ kk ]);
      }
      { jj in [0,uj],kk in [0,uk]:
          mipp::Reg<TYPE> b{{ kk }}{{ jj }};
          b_{kk,jj}.load(&B[(k+ kk )*BLOCKJ + j + jj*nv]);
      }    
      { ii in [0,ui], jj in [0,uj] :
          c_{ii,jj} = { k in [0,uk]: mipp::fmadd(a_ii,kk, b_kk,jj, } , c_ii,jj  { k in [0,uk]: ) };
      }
    }
    { ii in [0,ui],jj in [0,uj]:
        c_ii,jj.store(&C[(i+ ii )*BLOCKJ + j + ( jj )*nv]);      
    }
  }
}
  \end{lstlisting}
  \caption{Matrix product vectorized and unrolled\label{fig:gemm}. \texttt{in} constructs correspond to unrolled statements/expressions. }
\end{figure}


\subsection{Impact of unrolling factor on Intel CascadeLake}
\begin{figure*}[h!]
%  \begin{subfigure}[h]{0.45\textwidth}
%  \includegraphics[width=\textwidth]{../benches/gemm/cascadelake-64x256x64/icc-2021.2.0.pdf}
%  \caption{icc 2021.2.0}
%  \end{subfigure}
  \begin{subfigure}[h]{0.45\textwidth}  
\includegraphics[width=\textwidth]{../benches/gemm/cascadelake-64x256x64/icc-19.1.3.pdf}
  \caption{icc 19.1.3}
  \end{subfigure}
  \begin{subfigure}[h]{0.45\textwidth}  
\includegraphics[width=\textwidth]{../benches/gemm/cascadelake-64x256x64/pbqp.pdf}
  \caption{Clang 15 with PBQP register allocation}
  \end{subfigure}
  \begin{subfigure}[h]{0.45\textwidth}  
\includegraphics[width=\textwidth]{../benches/gemm/cascadelake-64x256x64/greedy.pdf}
  \caption{Clang 15 with greedy register allocation}
  \end{subfigure}
  \begin{subfigure}[h]{0.45\textwidth}  
\includegraphics[width=\textwidth]{../benches/gemm/cascadelake-64x256x64/basic.pdf}
  \caption{Clang 15 with basic register allocation}
  \end{subfigure}
  \begin{subfigure}[h]{0.45\textwidth}  
\includegraphics[width=\textwidth]{../benches/gemm/cascadelake-64x256x64/gcc-11.9.pdf}
  \caption{GCC 11.9}
  \end{subfigure}\hfill ~
  \caption{Block DGEMM Performance on Intel CascadeLake w.r.t. register and spill usage. Performance is given in \% of peak flops/cycle (here 32). The block (BI,BK,BJ) is 64x256x64. Each code represents a different register block size, obtained through the three unrolling factors. Codes with register block size $\leq 32$ are in blue (should fit in registers whatever the schedule), other codes are in grey (with spill) and red (without spill).  \label{fig:cascadelake}}
\end{figure*}
 For all plots in Fig.\ref{fig:cascadelake}, the same codes with the same vectorization expressed with MIPP are
compiled. All codes are tested 40 times, the best performance is shown. Only one memory allocation is used for all tests. Points shown in blue correspond to codes that should not require
more SIMD registers than the physical registers (32 here, shown by the vertical red line), whatever
the scheduling chosen by the compiler. This is not the case for codes
shown in red, some schedules may manage to use only registers while
others will require spill code. The graphs show performance w.r.t. the
cumulated register count and spill used by the code. For instance, a value of 40 registers and
spill used indicates that the compiler was able to use all 32
registers and 8 additional spaces on the stack for spill. With the exception of
GCC, other compilers manage to allocate less than 32 registers for all
blue codes. The max performance on this machine is normally 32 flops/cycles (with 2 FMAs operating on AVX512 vectors).  

There is no significant difference between icc version 2021 and Clang
15, either greedy or PBQP register allocations. This is expected since
icc now is based on LLVM. GCC however fails significantly, reaching
only 24 flop/cycle while other compilers reach 30-31 flop/cycle.  For
the following experiments, only Clang 15 with greedy register
allocation is considered.

We observe that the best performance is reached with Clang 15 and
icc-19.1.3 when the codes use exactly 32 registers. With gcc-11.9, the
best code using exactly 32 registers is among the best codes obtained
with this compiler.  Tables in Figure \ref{fig:tables} show the
performance of all codes using exactly 32 registers, no spill. The
block size refers to the size taken by the input matrices manipulated
in the inner loop, in number of vectors. Sizes larger than 32 imply that the compiler has rescheduled loads during the computation and reuses SIMD registers. We observe that the best codes are different from one compiler to the other. 

\begin{figure*}[h!]
% For clang 15
{    \scriptsize
    \begin{tabular}{|c|c|c|c|}
      \hline
      \multicolumn{4}{|c|}{Code performance with Clang} \\
      \hline
      Unrolling & flop/cycle & block  & ILP\\
      ixkxj& in peak \% & size & \\
      \hline
14x1x2 & 93.59 & 44 & 28 \\
4x1x6 & 93.45 & 34 & 24 \\
5x3x4 & 93.34 & 47 & 20 \\
6x1x4 & 93.78 & 34 & 24 \\
6x2x3 & 93.84 & 36 & 18 \\
6x2x4 & 93.95 & 44 & 24 \\
5x1x5 & 94.79 & 35 & 25 \\
6x3x3 & 94.47 & 45 & 18 \\
6x3x4 & 94.53 & 54 & 24 \\
6x4x3 & 94.05 & 54 & 18 \\
7x2x3 & 94.81 & 41 & 21 \\
8x1x3 & 94.65 & 35 & 24 \\
7x3x3 & 95.05 & 51 & 21 \\
8x2x3 & 95.12 & 46 & 24 \\
\hline
\end{tabular}
% For icc
    \begin{tabular}{|c|c|c|c|}
      \hline
      \multicolumn{4}{|c|}{Code performance with icc 19.1.3} \\
      \hline
      Unrolling & flop/cycle & block  & ILP\\
      ixkxj& in peak \% & size & \\
      \hline
4x3x3 & 78.09 & 33 & 12 \\
3x2x6 & 79.18 & 36 & 18 \\
3x2x8 & 79.61 & 46 & 24 \\
3x1x8 & 81.90 & 35 & 24 \\
4x3x4 & 81.55 & 40 & 16 \\
7x2x2 & 83.77 & 32 & 14 \\
4x2x4 & 84.94 & 32 & 16 \\
6x2x3 & 84.73 & 36 & 18 \\
5x2x4 & 87.42 & 38 & 20 \\
4x2x5 & 88.23 & 38 & 20 \\
4x1x6 & 89.93 & 34 & 24 \\
10x1x2 & 90.70 & 32 & 20 \\
5x1x5 & 93.48 & 35 & 25 \\
6x1x4 & 95.23 & 34 & 24 \\
\hline
    \end{tabular}
% For gcc
    \begin{tabular}{|c|c|c|c|}
      \hline
      \multicolumn{4}{|c|}{Code performance with gcc 11.9} \\
      \hline
      Unrolling & flop/cycle & block  & ILP\\
      ixkxj& in peak \% & size & \\
      \hline
2x5x4 & 54.09 & 38 & 8 \\
3x6x2 & 54.83 & 36 & 6 \\
2x3x5 & 55.79 & 31 & 10 \\
2x2x6 & 57.52 & 28 & 12 \\
3x3x5 & 57.37 & 39 & 15 \\
3x5x4 & 57.08 & 47 & 12 \\
3x2x5 & 59.60 & 31 & 15 \\
3x5x3 & 59.15 & 39 & 9 \\
2x4x4 & 60.30 & 32 & 8 \\
3x4x4 & 60.33 & 40 & 12 \\
3x5x2 & 60.89 & 31 & 6 \\
3x3x4 & 62.91 & 33 & 12 \\
3x4x3 & 62.37 & 33 & 9 \\
5x3x2 & 71.70 & 31 & 10 \\
\hline
    \end{tabular}
}
\caption{Performance of the best 14 codes with Clang15,  icc 19.1.3 and gcc 11.9. Only codes using 32 registers and 0 spill are considered here\label{fig:tables}. block size corresponds to $ui*uj+ui*uk+uk*uj$. $40$ means that the code requires $40$ zmm registers if all loads are scheduled before the computation. ILP corresponds here to $ui*uj$.}
\end{figure*}

\subsection{Impact of expected ILP on Intel CascadeLake}
Figure \ref{fig:cascadelakeilp} shows the performance according to the parallelism expected from unrolling, here $ui * uj$. Indeed, $uk$ does not create additionnal ILP. From this plot, we observe that for icc and clang, the best codes result from a high ILP. On the contrary, for gcc the code without spill with the highest expected ILP only reaches $42\%$ of the max performance. The maximum performance  

\begin{figure*}[h!]
%  \begin{subfigure}[h]{0.45\textwidth}
%  \includegraphics[width=\textwidth]{../benches/gemm/cascadelake-64x256x64/icc-2021.2.0.pdf}
%  \caption{icc 2021.2.0}
%  \end{subfigure}
  \begin{subfigure}[h]{0.45\textwidth}  
\includegraphics[width=\textwidth]{../benches/gemm/cascadelake-64x256x64/icc-19.1.3p.pdf}
  \caption{icc 19.1.3}
  \end{subfigure}
  \begin{subfigure}[h]{0.45\textwidth}  
\includegraphics[width=\textwidth]{../benches/gemm/cascadelake-64x256x64/greedyp.pdf}
  \caption{Clang 15}
  \end{subfigure}
  \begin{subfigure}[h]{0.45\textwidth}  
\includegraphics[width=\textwidth]{../benches/gemm/cascadelake-64x256x64/gcc-11.9p.pdf}
  \caption{GCC 11.9}
  \end{subfigure}\hfill ~
  \caption{Block DGEMM Performance used on Intel CascadeLake according to expected ILP. Performance is given in \% of peak flops/cycle (here 32). The expected ILP is the product of the unrolled factors $ui * uj$. The block (BI,BK,BJ) is 64x256x64. Each code represents a different register block size, obtained through the three unrolling factors. Codes with register block size $\leq 32$ are in blue (should fit in registers whatever the schedule), other codes are in grey (with spill) and red (without spill).\label{fig:cascadelakeilp}}
\end{figure*}



Figure \ref{fig:cascadelakecorrelation}  shows the correlation between the codes compiled by the three compilers. One dot corresponds to the same source code, with the same unrolling factors. This confirms that the correlation is rather weak, with the exception of the codes where the register block size corresponds to 32.  

\begin{figure*}[h!]
  \begin{subfigure}[h]{0.45\textwidth}
  \includegraphics[width=\textwidth]{../benches/gemm/cascadelake-64x256x64/icc21xicc19.pdf}
  \caption{Clang 15 compared to icc 19.1.3}
  \end{subfigure}
  \begin{subfigure}[h]{0.45\textwidth}  
\includegraphics[width=\textwidth]{../benches/gemm/cascadelake-64x256x64/gccxgreedy.pdf}
  \caption{Clang 15 compared to gcc 11.9}
  \end{subfigure}

  \caption{(a) and (b) Correlation of code performance on CascadeLake between different compilers.\label{fig:cascadelakecorrelation}. In (a), the red dots correspond to codes that have a register block size of exactly 32. It implies that whatever the scheduling, it fits into registers.}
\end{figure*}

Finally, Figure \ref{fig:cascadelakeloc} shows for a given number of
source code lines, the performance obtained. Surprisingly, performance
culminates with 90 loc and then it degrades when the number of source
code lines. Note that the unrolling factor on k, creating a long
sequence of FMAs, is assumed to be on one single line of code.
\begin{figure*}[h!]
  \includegraphics[width=0.45\textwidth]{../benches/gemm/cascadelake-64x256x64/icc21loc.pdf}
  \caption{Performance wrt lines of source code\label{fig:cascadelakeloc}}
\end{figure*}


\subsection{Impact of unrolling factor and SMT on Cavium Arm ThunderX2 and Intel CascadeLake}
The codes considered are exactly the same as before (portability).  Due to the architecture of the processor, supporting a SMT of 4 for the Cavium, and SMT 2 for the Intel, we compared performance for a single thread and performance with 4 and 2  threads resp., both on a single core. 

\begin{figure*}[h!]
  \begin{subfigure}[h]{0.45\textwidth}
  \includegraphics[width=\textwidth]{../benches/gemm/arm-64x256x64/greedy.pdf}
  \caption{clang 15}
  \end{subfigure}
  \begin{subfigure}[h]{0.45\textwidth}  
\includegraphics[width=\textwidth]{../benches/gemm/arm-64x256x64/openmp.pdf}
  \caption{clang 15 with OpenMP, SMT of 4}
  \end{subfigure}
    \caption{Block DGEMM Performance used on Cavium Arm ThunderX2 with (b) and without SMT (a). Performance is given in \% of peak flops/cycle (here 8). The block (BI,BK,BJ) is 64x128x64. Each code represents a different register block size, obtained through the three unrolling factors. Codes with register block size $\leq 32$ are in blue (fit in register whatever the schedule), others in red. \label{fig:arm}}
\end{figure*}

The correlation between performance with and without SMT, both for Cavium Arm ThunderX2 (clang 15) and Intel CascadeLake (icc 19.1.3) are shown in Fig.\ref{fig:SMT}. For Arm, this is for 4 threads, for Intel this is for 2 threads. On Arm, SMT brings a $8\%$ improvement on the best code, and on Intel, the same percentage applies. However, the performance have been degraded when compiling with openmp flag (and running with 1 thread) compared to the non-openmp version.
\begin{figure*}[h!]
  \begin{subfigure}[h]{0.45\textwidth}  
\includegraphics[width=\textwidth]{../benches/gemm/arm-64x256x64/clangxsmt.pdf}
  \caption{Correlation between Clang 15 without and with SMT4}
  \end{subfigure}
    \begin{subfigure}[h]{0.45\textwidth}  
\includegraphics[width=\textwidth]{../benches/gemm/cascadelake-64x256x64/icc1thx2th.pdf}
  \caption{Correlation between icc without and with SMT2}
    \end{subfigure}
    \caption{Correlation between performance without SMT and with SMT, on Cavium Arm ThunderX2 (SMT4) and on Intel CascadeLake (SMT2). Performance is given in \% of peak flops/cycle. \label{fig:SMT}}
\end{figure*}


\subsection{Comparing performance on AVX2, different microarchitectures}
The codes considered are exactly the same as before (portability). We evaluated the same code on AMD Zen 2, Zen 3 architectures and on Intel Haswell, all being AVX2 compatible architectures. Of course, the microarchitecture is different, and the compiler may choose to generate different codes. Only Clang 15 with greedy register allocation has been considered here. 

The experiments show that the best performing codes have the same unrolling factors and the assembly code generated are the same. 
\begin{figure*}[h!]
  \begin{subfigure}[h]{0.45\textwidth}
  \includegraphics[width=\textwidth]{../benches/gemm/haswell-64x256x64/greedy.pdf}
  \caption{Clang 15 on Intel Haswell}
  \end{subfigure}
  \begin{subfigure}[h]{0.45\textwidth}  
\includegraphics[width=\textwidth]{../benches/gemm/zen3-64x256x64/greedy.pdf}
  \caption{Clang 15 on AMD Zen3$^*$}
  \end{subfigure}
  \begin{subfigure}[h]{0.45\textwidth}  
\includegraphics[width=\textwidth]{../benches/gemm/zen3-64x256x64/greedy-zen2.pdf}
  \caption{Clang 15 on AMD Zen2$^*$}
  \end{subfigure}
  \caption{Block DGEMM Performance used on AVX2 architectures. Performance is given in \% of peak flops/cycle (here 8). The block (BI,BK,BJ) is 64x128x64. Each code represent a different register block size, obtained through the three unrolling factors. Codes with register block size $\leq 32$ are in blue (fit in register whatever the schedule), others in red. (c) show correlation between the two plots. $^*$ Performance scaled down to take into account turboboost effect, active on these machines\label{fig:avx2}}
\end{figure*}



\begin{figure*}[h!]

  \includegraphics[width=\textwidth]{../autobench/datas/axya/axya_ui_1_32_uk_1_32.pdf}
  \caption{Peformance of the sum of product $A\times x+y\times A$, with $A$ a $1024\times1024$ matrix and $x,y$ two vectors of length $1024$. Type of data is float.}
\end{figure*}


\section{Related Works}
\section{Conclusion}
\label{sec:conclusion}

\bibliography{biblio}

\end{document}
